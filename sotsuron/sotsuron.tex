\documentclass[uplatex, titlepage, dvipdfmx, 12pt, a4paper]{jsreport}
\usepackage{geometry}                		% See geometry.pdf to learn the layout options. There are lots.
\geometry{margin=2.5cm}                   		% ... or a4paper or a5paper or ... 
%\geometry{landscape}                		% Activate for rotated page geometry
\usepackage[parfill]{parskip}    		% Activate to begin paragraphs with an empty line rather than an indent
\usepackage[dvipdfmx]{graphicx}				% Use pdf, png, jpg, or eps§ with pdflatex; use eps in DVI mode		
\usepackage{amssymb}
\usepackage{siunitx}
\setcounter{tocdepth}{3}

\title{ \huge 卒業論文\\ \Huge リングイメージ型検出器(RICH)\\の性能評価}
\author{ \LARGE 戸田 匡哉 \and \LARGE 徳田 恵}
\date{\Large \today}							

\begin{document}
\maketitle
\begin{abstract}
 我々は J-PARCハドロン実験施設の高運動量ビームラインにおいてチャームバリオン分光実験 (J-PARC-E50) を計画している。実験では運動量20 GeV/{\sl c} のビーム粒子と標的との反応で生成する 2$\sim$16 GeV/{\sl c}の高運動量散乱粒子の識別を行う必要がある。この広い運動量領域において粒子識別を行うためにリングイメージチェレンコフ検出器を開発している。\\
 粒子識別性能に関わる角度分解能のうち、収差等の内訳を調べるため、テスト機を用いてSPring-8 のLEPSビームラインにおいてテスト実験を行った。光センサーとしてMPPCアレイを使用し、同時にMPPCの暗電流による角度分解能への影響や電圧依存性を調査した。MPPCアレイは8$\times$8の64セグメントで、各セグメントの大きさが3mm$\times$3mmのものを用いた。実験では空気を輻射体とし、0.95 GeV/{\sl c}の電子・陽電子からのチェレンコフ光を球面鏡によってMPPCアレイ上に結像させた。また、MPPCの動作電圧を0.5Vずつ54.0V-57.5Vの間で変化させ、取得したTDCデータからリングイメージを測定した。\\
 得られたリングイメージからリング中心を決定しイベント毎にチェレンコフ角を計算、角度分解能のhit数依存性から角度分解能をセグメントサイズによる位置分解能$\Delta\theta_{seg}$と、収差などによる分解能$\Delta\theta_{other}$に分けた。セグメントサイズによる位置分解能は用いたMPPCの1セグメントの大きさと球面鏡の焦点距離から計算によって求められ、$\Delta\theta_{seg}=2.80$ mradとなる。収差などによる分解能は、角度分解能のhit数依存性のグラフからフィッティングから見積もり、$\Delta\theta_{other}=3.88 \pm 0.02$ mradとなった。ここから焦点面と検出面の位置のずれや、入射粒子の角度分解能、色収差などに細分し、残ったものを暗電流の影響とした。\\
 また各動作電圧でのイベントごとのhit数から、hit数の電圧依存性とシミュレーションによる再現を行なった。


\end{abstract}
\tableofcontents
\chapter{序論}
\section{チャームバリオン分光実験(E50実験)}
 現在、J-PARCではダイクォーク相関の解明を目的としたチャームバリオン分光実験、E50実験が計画されている。この実験ではビーム運動量20 GeV/{\sl c}の$\pi^-$を液体水素標的に当て、次のチャームバリオン$Y^{*-}_c$生成反応を起こす。
\begin{equation}
\pi^- + p \to Y^{*-}_c + D^{*-}
\end{equation}
この時に生じる$D^{*-}$の崩壊モード
\begin{equation}
D^{*-} \to \overline{D}^0\pi^- \to K^+\pi^-\pi^-
\end{equation}
ここから生じる$K^+, \pi^-, \pi^-$の運動量から$D^{*-}$を再構成し、Missing Mass法によって$Y^{*-}_c$を測定する。

\begin{figure}[htbp]
  \begin{center} 
    \includegraphics[clip, scale=0.8]{image/E50setup.png}
    \caption{E50実験のセットアップ。図後方のものがリングイメージチェレンコフ(RICH)検出器} 
    \label{fig:E50setup} 
  \end{center}
\end{figure}

$D^{*-}$の一連の崩壊から生じる$K^+, \pi^-$を2$\sim$16 GeV/{\sl c}の広い運動量領域で粒子識別を行う必要がある。そこで用いられるPID検出器が図\ref{fig:E50setup}後方のリングイメージチェレンコフ検出器である。

\section{測定原理}
 荷電粒子が物質中を通過する際にその速度が物質中の光速より早い場合、すなわち次式のような条件を満たす時、円錐状に光が発生する。
\begin{equation}
v > \frac{c}{n} \Leftrightarrow \beta > \frac{1}{n} 
\end{equation}
発生するチェレンコフ光と荷電粒子の進行方向がなす角をチェレンコフ角$\theta_c$と呼び、チェレンコフ角と荷電粒子の速度、物質の屈折率の関係は次式で表される。
\begin{equation}
\cos \theta_c = \frac{1}{n\beta}
\end{equation}
\begin{figure}[h]
  \begin{center} 
    \includegraphics[clip, scale=0.5]{image/cherenkov.png}
    \caption{チェレンコフ光が発生する様子} 
    \label{fig:cherenkov} 
  \end{center}
\end{figure}

リングイメージチェレンコフ検出器では測定したリングイメージの半径と、輻射体から光センサーまでの距離$L$を用いて次の式からチェレンコフ角を算出する。
\begin{equation}
\tan \theta_c = \frac{r}{L} \Leftrightarrow \theta_c = \arctan \left(\frac{r}{L}\right)
\end{equation}
このチェレンコフ角を用いて式(1.4)より、粒子の速度を求めることで粒子識別を行う。
\begin{figure}[htbp]
  \begin{center}
    \begin{tabular}{c}

      % 1
      \begin{minipage}{0.33\hsize}
        \begin{center}
          \includegraphics[clip, scale=0.6]{image/pion.png}
          \hspace{1.6cm}
        \end{center}
      \end{minipage}

      % 2
      \begin{minipage}{0.33\hsize}
        \begin{center}
          \includegraphics[clip, scale=0.6]{image/kaon.png}
          \hspace{1.6cm}
        \end{center}
      \end{minipage}

    \end{tabular}
    \caption{$\pi^-とK^+によるチェレンコフ光$}
    \label{fig:piKcherenkov}
  \end{center}
\end{figure}

\section{研究目的}
\subsection{先行研究によるRICH検出器実機の構成と要求性能}
 先行研究での実機の構成として、エアロゲル(n=1.04)と$\rm{C_4F_{10}}$ガス(n=1.0037)の2種類の輻射体を用いて、広い運動量領域においてチェレンコフ光を放出させる。このチェレンコフ光を球面鏡で反射させ、光センサー上で収束させることでリングイメージを観測する。

\begin{figure}[htbp]
  \begin{center} 
    \includegraphics[clip, scale=0.8]{image/RICH.png}
    \caption{実機のデザイン(先行研究より)} 
    \label{fig:RICH} 
  \end{center}
\end{figure}
 実機の要求性能は次のようになっている。目標とする粒子識別性能99\%に要求される角度分解能は$\Delta\theta_c < 10 \rm{mrad}$であり、角度分解能は以下の要素に影響される。
\begin{enumerate}
  \item セグメントサイズの大きさによる位置分解能
  \item 色収差、輻射体の厚さによる分散、検出面での収束等の収差などの影響
  \item ビームの角度分解能
\end{enumerate}
このうち、収差などの影響の内訳は詳細にわかっていないため、球面鏡と光センサーとしてMPPCを用いた小型のテスト機を用いて収差の内訳について調査を行なった。
\subsection{Multi Pixel Photon Counter}
 今回のテスト実験で用いた光センサーであるMulti Pixel Photon Counter、MPPCは安価である、磁場に強いといった利点がある。一方で熱電子が増幅され、1光電子が常にノイズとして出ており、暗電流が大きいという特徴がある。その計数率は100~300kHzでありRICH検出器では1光電子を捉えるため、その影響についても併せて調査しシミュレーションによる再現を行なった。また電圧によってゲインと検出効率が変わるといった特徴もあり、以降ではMPPCの動作電圧$V_{operation}$から、信号が出始める電圧$V_{breakdown}\sim51V$を引いた$V_{over voltage}$という表記を用いる。
\chapter{実験のセットアップ}
  2020年12月にSPring-8のLEPSビームラインにて実験を行なった。0.95 GeV/$c$の電子・陽電子をビームとして用い、空気を輻射体とした。暗箱内にはビーム軸に対してチェレンコフ光が$15^\circ$の角度で反射するように球面鏡を配置し、MPPC上に収束させた。暗箱外側の先端と後部にはトリガーカウンターとして10mm角のプラスチックシンチレーターをそれぞれビーム軸上に配置した。
 
 \begin{figure}[htbp]
  \begin{center} 
    \includegraphics[clip, scale=0.6]{image/setup.png}
    \caption{実験のセットアップ}
    \label{fig:setup} 
  \end{center}
\end{figure}
MPPCは$8\times8$の2次元アレイ、64セグメントのもの使用し、1セグメントの大きさは3.2mm角である。動作電圧は54~57.5Vまで0.5V刻みで測定を行なった。読み出しにはNIM-EASIROC moduleを使用し、TDC情報を取得した。使用した球面鏡は直径が150mm、曲率半径は660mm、焦点距離はその半分の330mmである。

\begin{figure}[htbp]
  \begin{center}
    \begin{tabular}{c}

      % 1
      \begin{minipage}{0.33\hsize}
        \begin{center}
          \includegraphics[clip, scale=0.6]{image/MPPC.png}
          \hspace{1.6cm} [1] MPPC
        \end{center}
      \end{minipage}

      % 2
      \begin{minipage}{0.33\hsize}
        \begin{center}
          \includegraphics[clip, scale=0.6]{image/mirror.png}
          \hspace{1.6cm} [2] 球面鏡
        \end{center}
      \end{minipage}

    \end{tabular}
    \caption{使用したMPPCと球面鏡}
    \label{fig:MPPC'N'mirror}
  \end{center}
\end{figure}


\chapter{解析}
  \section{TDC}
   TDCを取得するための閾値は動作電圧が最も低い$V_{ov}=3.0V$の時の1p.e.にかけた。取得したTDCのヒストグラムが図\ref{fig:tdc}である。
  \begin{figure}[htbp]
    \begin{center} 
      \includegraphics[clip, scale=0.3]{image/tdc.png}
      \caption{得られたTDCのヒストグラム。全てのチャンネルを足し合わせている。}
      \label{fig:tdc} 
    \end{center}
  \end{figure}
  図\ref{fig:tdc}より、今回のTDCcut幅は20と決定した。また、幅は20に固定し、cut positionは$\frac{signal}{bg}$比が最も良くなるところをscanによって決定した。
  \begin{figure}[htbp]
    \begin{center} 
      \includegraphics[clip, scale=0.3]{image/TDCcut_scan.pdf}
      \caption{Vov 5\si{V}でのscanの様子。横軸はcutのスタート位置でcut幅は20で固定している。}
      \label{fig:tdc_scan} 
    \end{center}
  \end{figure}
  図\ref{fig:tdc_scan}より、今回のTDCcut positionは$287\sim307 \si{ch}$に決定した。

  \section{hit数}
    先ほど決定したTDCcutを用いて、1イベント当たりに光子が入ってきたセグメントの数を数え、それをhit数とした。
    ただし、今回はTDCでhitの有無のみをみているため、1セグメントに光子が複数個入っても1hitとして数える。
    \begin{figure}[h]
      \begin{center} 
        \includegraphics[scale=0.5, clip]{image/hit_image.pdf}
        \caption{あるイベントで黄色のセグメントに光子が入ってきた場合の図。この場合だと10hitとなる。} 
        \label{fig:hit_image} 
      \end{center}
    \end{figure}
    
    全イベントでhit数を出し、そのヒストグラムをガウスフィットすることでピークから各電圧での平均のhit数を決定した。
    また、TDCのcut範囲をピーク位置($287\sim307\si{ch}$)からずらした$400\sim420\si{ch}$にしたときのhit数を暗電流によるhit数とした。
    
    \begin{figure}[h]
      \begin{center} 
        \includegraphics[scale=0.5, clip]{image/Vov5_nhits.pdf}
        \caption{Vov$\SI{5}{V}$のnhits} 
        \label{fig:nhits} 
      \end{center}
    \end{figure}

  \section{hit patternの確認}
    先ほど求めたhit数をセグメントごとに全イベント分足し上げることで、hit patternを確認したところ、図\ref{fig:hitpattern}のようにチェレンコフリングを確認することができた。
    \begin{figure}[h]
      \begin{center} 
        \includegraphics[scale=0.5, clip]{image/hitpattern.pdf}
        \caption{Vov$\SI{5}{V}$のhitpattern} 
        \label{fig:hitpattern} 
      \end{center}
    \end{figure}

  \section{リングイメージの中心}
    hit patternを見るとわかるように、今回の実験ではアレイの中心とリングの中心がズレていることがわかった。
    そこで、リング中心の決定を最初におこなった。
    \subsection{決定方法1}
      まずアレイを1列ずつ取り出し、円周上に来るところで2つのピークが見えるのでそのピーク間の距離を求めた。
      円の方程式より、図\ref{fig:find_center_image}の$x_1. x_2, y_0$には次のような関係が成り立つ。
      \begin{equation}
        x_2 - x_1 = 2\sqrt{{p_0}^2-{\left(y_0-p_1\right)}^2}
        \label{circle_equation}
      \end{equation}
      求められたピーク間の距離を式\ref{circle_equation}でフィッティングすることで円の中心を求めた。
      \begin{figure}[h]
        \begin{center} 
          \includegraphics[scale=1, clip]{image/find_center1.pdf}
          \caption{yを固定した時の図。$x_1, x_2$の2点のピークが見えるので、ピークの間隔から$x_2-x_1$を求める。} 
          \label{fig:find_center_image} 
        \end{center}
      \end{figure}
      \begin{figure}[h]
        \begin{tabular}{cc}
          %---- 最初の図 ---------------------------
          \begin{minipage}[t]{0.45\hsize}
            \centering
            \includegraphics[scale=0.4, clip]{image/find_ycenter.pdf}
            \caption{ある$y$で切った時の$x_1,x_2$の差} 
            \label{fig:find_ycenter} 
          \end{minipage} &
          %---- 2番目の図 --------------------------
          \begin{minipage}[t]{0.45\hsize}
            \centering
            \includegraphics[scale=0.4, clip]{image/find_xcenter.pdf}
            \caption{ある$x$で切った時の$y_1,y_2$の差} 
            \label{fig:find_xcenter1} 
          \end{minipage}
          %---- 図はここまで ----------------------
        \end{tabular}
      \end{figure}
      この方法では、リングの中心はアレイのマス目を単位として$\left(3.3, 4.1\right)$と決定された。
    \subsection{決定方法2}
      ここにはリング中心の求め方の二つ目を書く

  \section{リング半径}
    イベントごとに、リング中心からhitしたセグメントまでの距離を平均し、それを全イベントで積み上げてヒストグラムを描いた。そのヒストグラムをガウスフィットし、meanをその電圧での平均半径とした。
    \begin{figure}[h]
      \begin{center} 
        \includegraphics[scale=0.5, clip]{image/radius.pdf}
        \caption{イベントごとに計算したリング半径を全イベント分で積み上げたヒストグラム} 
        \label{fig:5Vradius} 
      \end{center}
    \end{figure}
    
    
    
    \section{チェレンコフ角・角度分解能}
      チェレンコフ光の式より、チェレンコフ角$\theta_{c}$は、鏡の焦点距離$f$とリング半径$r$を用いて次のように表せる。
      \begin{equation}
        \theta_{c} = \arctan \left(\frac{r}{f} \right)
        \label{theta_radius}
      \end{equation}
      この式より、イベントごとのチェレンコフ角$\theta_{c}$を計算し、全イベント分を積み上げたヒストグラムをガウスフィットし、meanをその電圧でのチェレンコフ角とした。

      図(Vov5Vの時のチェレンコフ角)\\
      チェレンコフ角のヒストグラムをガウスフィットし、meanをその電圧でのチェレンコフ角$\theta_{c}$、分散を角度分解能とした。

\chapter{結果}
  \section{hit数}
    全電圧でのhit数をまとめると、図\ref{fig:hit_dark}の様にVov5Vでサチュレーションしていることがわかる。そこで、このサチュレーションしているhit数がもっともらしいことを計算により確かめる。
    チェレンコフ光の発生光子数$N_{0}$は、次の式で求めることができる。
    \begin{equation}
        N_{0} = 2 \pi \alpha L  \left(1 - \frac{1}{(n\beta)^2}\right) \left(\frac{1}{\lambda_{1}} - \frac{1}{\lambda_{2}}\right)
    \end{equation}
    ここで、$\alpha$は微細構造定数、$n$は輻射体の屈折率、$\beta$は粒子の速度、$L$は輻射体の長さ、$\lambda_{1}$,$\lambda_{2}$はMPPCの感度のある波長である。
    ただし、実際に検出される光子数$N_{det}$はこの$N_{0}$に鏡の反射効率$\epsilon_{mirror}$やMPPCの量子効率$\epsilon_{MPPC}$を用いて次のようにかける。
    \begin{equation}
        N_{det} = \int^{\lambda_2}_{\lambda_1} N_{0}\left(\lambda\right) \epsilon_{mirror}\left(\lambda\right) \epsilon_{MPPC}\left(\lambda\right) d\lambda
        \label{hit_count}
    \end{equation}
    また、今回は1つのセグメントに同時に複数個の光子が入ってきても1hitと数えることにより数え落としてしまうことがある。
    式\ref{hit_count}より検出光子数を計算し、シミュレーションによって消えてしまう光子数を見積もった。
    その結果、図\ref{fig:hit_simulation}のように、サチュレーションしているところでは解析値と一致していることがわかった。
    \begin{figure}[h]
      \begin{tabular}{cc}
        %---- 最初の図 ---------------------------
        \begin{minipage}[t]{0.45\hsize}
          \centering
          \includegraphics[scale=0.4, clip]{image/hit_count.pdf}
          \caption{信号と暗電流のhit数} 
          \label{fig:hit_dark} 
        \end{minipage} &
        %---- 2番目の図 --------------------------
        \begin{minipage}[t]{0.45\hsize}
          \centering
          \includegraphics[scale=0.35, clip]{image/hit_simulation.pdf}
          \caption{信号とsimulationのhit数} 
          \label{fig:hit_simulation} 
        \end{minipage}
        %---- 図はここまで ----------------------
      \end{tabular}
    \end{figure}
    
    % \end{figure}

  \section{リング半径・チェレンコフ角}
    各電圧でのチェレンコフ角とリング半径は図$\ref{fig:all_theta}, \ref{fig:all_radius}$のようになった。
    \begin{figure}[h]
      \begin{tabular}{cc}
        %---- 最初の図 ---------------------------
        \begin{minipage}[t]{0.45\hsize}
          \centering
          \includegraphics[scale=0.4, clip]{image/theta_allV.pdf}
          \caption{各電圧でのチェレンコフ角} 
          \label{fig:all_theta} 
        \end{minipage} &
        %---- 2番目の図 --------------------------
        \begin{minipage}[t]{0.45\hsize}
          \centering
          \includegraphics[scale=0.3, clip]{image/all_radius.pdf}
          \caption{各電圧でのリング半径} 
          \label{fig:all_radius} 
        \end{minipage}
        %---- 図はここまで ----------------------
      \end{tabular}
    \end{figure}


    \begin{table}[b]
      \begin{center}
        \label{tab:radius}
        \caption{リング半径の広がり (mm)}
        \begin{tabular}{|c|c|c|}\hline
          min & avg & xax\\ \hline
          4.54 & 6.62 & 8.69\\ \hline
        \end{tabular}
      \end{center}
    \end{table}
    \subsection{リング半径・チェレンコフ角の絶対値のズレ}
      まず、得られたチェレンコフ角・リング半径の絶対値について考える。
      ただし、チェレンコフ角とリング半径は式\ref{theta_radius}より1対1対応なので、ここではリング半径について考える。
      MPPCが球面鏡の焦点面に配置されていた場合、MPPC上でのリング半径$r$は球面鏡の焦点距離$f=330 mm$、チェレンコフ角$\theta_{c}$から$r=f\tan{\theta_{c}}$とかけ、計算すると$r=7.7 mm$となった。この値は、解析で得られた値より10\%ほど大きい。今回の実験ではMPPCの配置が球面鏡の焦点面から42 mmほどズレていたので、それが原因ではないかと考えた。
      図\ref{fig:optical_system}のように、MPPCが焦点面からズレていた場合焦点面で収束されるはずの光が内側と外側に広がってしまう。上流側の光が多いほど内側にくる光が多くなる。今回のセットアップによるズレを考慮して計算を行うと表$\ref{tab:radius}$のようになった。
      この平均値が解析で得られる値と考えると、先ほどとは逆に解析値の方が5\%ほど大きくなっている。
      \begin{figure}[h]
        \begin{center} 
          \includegraphics[scale=1, clip]{image/optical_system.png}
          \caption{光学系} 
          \label{fig:optical_system} 
        \end{center}
      \end{figure}
      これは絶対値のズレは暗電流が影響していると考えられる。
      \begin{figure}[h]
        \begin{center} 
          \includegraphics[scale=0.4, clip]{image/dark_current_image.pdf}
          \caption{黄:実際に光子が当たったセグメント、青:暗電流によるhitがあったセグメント} 
          \label{fig:darkcurrent_image} 
        \end{center}
      \end{figure}
      図$\ref{fig:darkcurrent_image}$のように、暗電流がリングの外側に出てしまった場合そのイベントのリング半径は実際よりも大きくなってしまう。
      リングの内側に暗電流が発生した場合はリング半径は小さくなるが、今回のリングとMPPCアレイのサイズ関係だと内側よりも外側の方が面積が大きいため、暗電流によってリング半径が大きく見積もられてしまう。
    \subsection{リング半径・チェレンコフ角の電圧依存性}
    リング半径とチェレンコフ角の電圧依存性を考えるために、簡単なシミュレーションをおこなった。
    リング中心は(3.3, 3.9)、リング半径は$4.54\sim\SI{8.69}{mm}$の範囲でランダムに決定。
    1イベントごとの発生光子数は68個、MPPCの検出確率は各電圧での値を用いた。
    また、1セグメントに複数の光子が同時に入る効果やセグメント間の隙間も考慮した。
    以上の条件で、実際の解析と同様の手順でリング半径を求めた。
    \begin{figure}[h]
      \begin{center} 
        \includegraphics[scale=0.4, clip]{image/radius_simulation.pdf}
        \caption{シミュレーションによるリング半径} 
        \label{fig:radius_simulation} 
      \end{center}
    \end{figure}
    図$\ref{figure:radius_simulation}$を見ると、シミュレーションでもリング半径に電圧依存性が見られることがわかる。
    これは暗電流の効果が原因と考えられる。
    暗電流にはリング半径を大きくする効果があり、電圧を上げていくと暗電流が増加していくので、リング半径・チェレンコフ角が増加していく。

  \section{角度分解能}
    各電圧での角度分解能をまとめると図$\ref{fig:resolution_all}$のようになった。
    \begin{figure}[h]
      \begin{center} 
        \includegraphics[scale=0.4, clip]{image/resolutin_allV.pdf}
        \caption{各電圧での角度分解能} 
        \label{fig:resolution_all} 
      \end{center}
    \end{figure}
    分解能もVov $\SI{5}{V}$あたりでサチュレーションしていることがわかる。
    \subsection{1 p.e.あたりの分解能}
    1 p.e.あたりの分解能を求めるために、イベントごとのhit数で場合分けをし分解能を求めた。
    図\ref{fig:per_hit}のグラフを1 p.e.あたりの分解能を$\Delta \theta_{1 p.e.}$として、$\frac{\Delta \theta_{1 p.e.}}{\sqrt{nhits}}$でフィットした。
    フィッティングの値より、$\Delta \theta_{1 p.e.} = 4.85 \pm 0.01 \: \si{mrad}$と求められた。
    \begin{figure}[h]
      \begin{center} 
        \includegraphics[scale=0.4, clip]{image/per_hit.pdf}
        \caption{hit数ごとの分解能} 
        \label{fig:per_hit} 
      \end{center}
    \end{figure}
    
    
    
    \subsection{角度分解能の内訳}
    
    \subsection{収差などによる角度分解能}
    
    \subsection{暗電流の角度分解能への寄与}
    暗電流の角度分解能への影響を調べるために、暗電流がある場合と、暗電流なしの極限(TDCのcut幅を0にした極限)での角度分解能と1 p.e.あたりの分解能をそれぞれ調べた。
    まず、TDCのcut幅を100 chまで20 chずつ広げていき暗電流の割合を増加させた時の分解能を調べることで、暗電流なしの極限での角度分解能を求めた。
    図\ref{fig:reso_tdc}に外挿して、cut幅0で暗電流なしの極限を見ると$\sim 1.38 \si{mrad}$の分解能を得られる。
    また、角度分解能と1 p.e.あたりの分解能の関係式$\frac{\Delta \theta_{1 p.e.}}{\sqrt{nhits}}=\Delta \theta$より、cut幅20の場合から
    \begin{eqnarray}
      \frac{4.85 \si{mrad}}{\sqrt{nhits}}&=&1.45 \si{mrad} \nonumber \\
      \sqrt{nhits} &=& 3.34 \nonumber \\
    \end{eqnarray}
    が求まる。
    よって、cut幅0での1 p.e.あたりの分解能$\Delta \theta_{1 p.e.}$は$\frac{\Delta \theta_{1 p.e.}}{\sqrt{nhits}}=1.38 \si{mrad}$より
    $\Delta\theta_{1 p.e.}=4.62 \si{mrad}$となる。
    \begin{figure}[h]
      \begin{center} 
        \includegraphics[scale=0.4, clip]{image/reso_tdc.pdf}
        \caption{TDCのcut幅を広げていった時の分解能} 
        \label{fig:reso_tdc} 
      \end{center}
    \end{figure}



\chapter{まとめ}
\end{document}  